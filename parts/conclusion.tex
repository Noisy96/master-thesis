\chapter{Conclusion \& perspectives}

The idea of adapting digital voting systems to make electoral processes cheaper, faster and easier, is a compelling
one in modern society. Making the electoral process cheap
and quick, normalizes it in the eyes of the voters, removes a
certain power barrier between the voter. It also opens the door for a more direct form of democracy, allowing voters to express their will on individual bills and propositions.

In this paper, we introduced a unique, blockchain-based electronic voting system by building a proof-of-concept that utilizes smart contracts to enable secure and cost efficient election while guaranteeing transparency and integrity. We have outlined the systems architecture, the design, and a visual demonstration of what it would look like. By comparison to previous work, we have shown that the blockchain technology offers a new possibility for democratic countries to advance from the pen and paper election scheme, to a more cost- and time-efficient election scheme, while increasing the security measures of the todays scheme and offer new possibilities of transparency. Using an Ethereum private blockchain, it is possible to send hundreds of transactions per second onto the blockchain, utilizing every aspect of the smart contract to ease the load on the blockchain. For countries of greater size, some measures must be taken to withhold greater throughput of transactions per second, for example the parent \& child architecture which reduces the number of transactions stored on the blockchain at a 1:100 ratio without compromising the networks security. Our election scheme allows individual voters to vote at a voting district of their choosing while guaranteeing that each individual voters vote is counted from the correct district, which could potentially increase voter turnout.

%\subsubsection{Ecological aspects}
%\subsubsection{Financial aspects}
%\subsubsection{Conclusion}
%\subsubsection{Perspectives}