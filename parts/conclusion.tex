\chapter{Conclusion \& perspectives}

The idea of adapting digital voting systems to make electoral processes cheaper, faster and easier, is a compelling one in modern society. Making the electoral process cheap and quick, normalizes it in the eyes of the voters, removes a certain power barrier between the voter and authorities. It also opens the door for developing countries to aspire for better democracy.

In this paper, we introduced a unique, blockchain-based electronic voting system by building a proof-of-concept that utilizes smart contracts to enable secure and cost efficient election while guaranteeing transparency and integrity. We have outlined the systems architecture, the design, and a visual demonstration of what it would look like. Essential trade-offs had to be made along the way, for example introducing a complementary backend server that serves the purpose of cutting the storing of data on the Blockchain costs significantly, however it also exposes the system to DDoS attacks that can target the authentication phase of the voting and jeopardize the whole election. One possible solution to remedy that particular issue would be the use of local databases on the voting machines and performing authentication locally on each machine, that would not eliminate all the risks, however it would ensure that the election can't be stopped by malicious third parties.

Ethereum Blockchain technology as it stands today suffers various shortcomings, the technology is vastly criticized for its outrageous energy consumption and carbon footprint, to put it in perspective a typical Ethereum transaction gobbles more power than an average U.S. household uses in a day\cite{fairleyEthereumWillCut2019}, but the Ethereum founders and community are promising a more scalable, more secure and more sustainable version of the technology that they're calling Ethereum 2.0, it uses 1\% of the energy it uses today to complete transactions\cite{fairleyEthereumWillCut2019}.

For the future, it will be interesting to see how the upgrades on Ethereum could impact this particular application of the technology, or will there ever be a new Blockchain implementation tailored specifically for elections.

Our system design and prototype remain to be tested on real Ethereum \gls{testnets} and further optimize our system.