\makeglossaries
 
\newglossaryentry{cryptocurrency}
{
    name=cryptocurrency,
    description={A cryptocurrency, crypto-currency, or crypto is a digital asset designed to work as a medium of exchange wherein individual coin ownership records are stored in a ledger existing in a form of a computerized database using strong cryptography to secure transaction records}
}

\newglossaryentry{51attack}
{
    name=51\% attack,
    description={A 51\% attack refers to an attack on a Proof-of-Work (PoW) blockchain where an attacker or a group of attackers gain control of 51\% or more of the computing power or hash rate. PoW is a system of consensus used by blockchains to validate transactions}
}

\newglossaryentry{ETH}
{
    name=ETH,
    description={or Ether is a cryptocurrency. It is scarce digital money that you can use on the internet, It's the currency of Ethereum apps}
}

\newglossaryentry{cryptoeconomic}
{
    name=Cryptoeconomics,
    description={Cryptoeconomics describes an interdisciplinary, emergent and experimental field that draws on ideas and concepts from economics, game theory and related disciplines in the design of peer-to-peer cryptographic systems.}
}

\newglossaryentry{proof-of-concept}
{
    name=Proof-of-concept,
    description={also known as proof of principle, is a realization of a certain method or idea in order to demonstrate its feasibility, or a demonstration in principle with the aim of verifying that some concept or theory has practical potential. A proof of concept is usually small and may or may not be complete}
}

\newglossaryentry{DDoS}
{
    name=DDoS attack,
    description={In computing, a denial-of-service attack is a cyber-attack in which the perpetrator seeks to make a machine or network resource unavailable to its intended users by temporarily or indefinitely disrupting services of a host connected to the Internet}
}

\newglossaryentry{backend}
{
    name=Backend,
    description={The backend (or “server-side”) is the portion of the website the end user doesn't see. It's responsible for storing and organizing data, and ensuring everything on the client-side actually works. The backend communicates with the front-end, sending and receiving information to be displayed as a web page}
}